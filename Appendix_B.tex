\chapter{C�digo para Inser��o do Ru�do �ptico Atrav�s do M�todo de
Imposi��o}
\label{AppendixB}

% This LaTeX was auto-generated from MATLAB code.
% To make changes, update the MATLAB code and republish this document.

% \documentclass{article}
% \usepackage{graphicx}
% \usepackage{color}

\sloppy
\definecolor{lightgray}{gray}{0.5}
\setlength{\parindent}{0pt}

% \begin{document}

    
\begin{verbatim}
%
function sinal_ruidoso = imposicao_ruido(sinal,Bref,Bo,OSNR_dB);
%==========================================================================
% imposicao_ruido - Fun��o para inserir ru�do �ptico conforme m�todo de
%                   imposi��o a partir da OSNR fornecida.
%
%          Chama as fun��es do Matlab:
%               mean
%               abs
%               randn
%               sqrt
%
%  by Thiago Machado
%  em $29/04/2013$
% =========================================================================

% ---- Insere o ru�do confome um valor de OSNR fornecido ------------------
Ps   = mean(mean(abs(sinal).^2));  % Potencia do sinal �ptico)
osnr = 10^(OSNR_dB/10);            % OSNR na escala linear
No   = Ps/(Bo*osnr);               % Determina a PSD do ru�do

% Gera o Ru�do Normalizado
ruido_norm = sqrt(1/2)*(randn(size(sinal))+ j*randn(size(sinal)));

% Potencializa o Ru�do Gerado
Pn    = sqrt(No*Bref);    % Potencia do ru�do
ruido = Pn*ruido_norm;    % Potencializa o ru�do

% Adiciona o Ru�do
sinal_ruidoso = sinal_ruidoso + ruido;

% -------------------------- Fim -----------------------------------------
\end{verbatim}

%         \color{lightgray} \begin{verbatim}Error using OSNR (line 18)
% Not enough input arguments.
% \end{verbatim} \color{black}
    


% \end{document}
    
