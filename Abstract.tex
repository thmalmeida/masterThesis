\begin{resumo}[Abstract]
 \begin{otherlanguage*}{english}

The orthogonal frequency division multiplexing OFDM techique applied to the
high rate optical communication transmission systems has become interesting
mainly for its ability to electroincally compensate for the effects of
chromatic dispersion CD and polarization mode dispersion PMD throught long
range links. However, insertion of OFDM signals inside intensity modulation
direct detection links requires the use of a band guard $B_G$ to avoid
intermodulation products caused by the beating between the light source and the
photodiode therefore the setting of various optical and electrical parameters
for successfully transmission.
 
This work consists in develop one experimental optical system of intensity
modulation and direct detection to apply OFDM signals into two different
DDO-OFDM modes of transmission. As also, characterizations of all optical and
electrical devices inserted into the communication system proposed are made, in
addition, the domain over the A/D and D/A high performance devices in order to
obtain better results during the step tests. Furthermore, due to innumerable
amount of design parameters involved in the signal construction's, most of then
are nonlinearly connected, makes this masther thesis propose an mono-objective
optimization of some parameters through genetic algorithm for the purpose of
take a better spectrally efficience and high data rates reducing the guard band
$B_G$ and the guard interval $I_G$.

\vspace{\onelineskip}
\noindent 
\textbf{Key-words}: IM/DD, DDO-OFDM, Optical OFDM, Experimental, Genetic
Algorithm, Mono-objective.
 \end{otherlanguage*}
\end{resumo}