\begin{resumo}[Abstract]
 \begin{otherlanguage*}{english}

The orthogonal frequency division multiplexing OFDM techique applied to the high
rate optical communication transmission systems has become interesting mainly
for its ability to electroincally compensate for the effects of chromatic
dispersion CD and polarization mode dispersion PMD throught long range links
standard single mode fiber SSMF. However, the insertion of OFDM signals in links
with intensity modulation and direct detection IMDD requires the use of a guard
band between the optical carrier and the band signal, to avoid interference from
intermodulation products generated by beats present in the direct detection of
the optical signal. Using a bandwidth equal to the bandwidth signal this guard
band significantly reduces optical system direct detection OFDM efficiency. This
Master Thesis is to improve the spectral efficiency of such systems by applying
a heuristic optimization parameters essential to the performance of the system
that seeks to minimize the aforementioned guard band without compromising system
performance as a whole. The suggested Genetic algorithm optimization allowed to
save of up to 15\% (0.3 GHz for a total of 2.0 GHz) on the total system
bandwidth. The transmission of OFDM signals in an experimental DDO-OFDM setup
implemented allowed the validation of optimization proposed.

\vspace{\onelineskip}
\noindent 
\textbf{Key-words}: Optical OFDM, direct detection, mathematical
optimization, Genetic algorithm, spectral efficiency.
 \end{otherlanguage*}
\end{resumo}
